\documentclass[12pt]{article}

%===============================
%
%          📦 Paquetes
%
%===============================

\usepackage[a4paper, top=2cm, bottom=2cm, left=2.5cm, right=2.5cm]{geometry}
\usepackage[spanish]{babel}
\usepackage[utf8]{inputenc}
\usepackage{amsmath}
\usepackage{multicol}
\usepackage{graphicx}
\usepackage{hyperref}
\usepackage{booktabs}
\usepackage{pgfplots}
\pgfplotsset{compat=1.18}
\usepackage{tikz}

\title{
  \vspace{2cm}
  \pagenumbering{gobble}
  \includegraphics[width=5cm]{./assets/logo-utp.png} \\
  \vspace{1cm}
  \textbf{Universidad Tecnológica del Perú} \\
  \vspace{2cm}
  \textbf{Cálculo I} \\
  \vspace{1cm}
  \large \textbf{Taller 1}
}
\author{
  \textbf{Torres Vara, Mateo Nicolas} - \texttt{U24308542} \\
  \texttt{Sección 32384}
}



\begin{document}
\maketitle
\begin{center}

  Docente: Victor Johnny Papuico Bernardo

\end{center}

%======================================
%
%          📚 Inicio del documento
%
%======================================

\newpage
\section*{Ejercicio 2}
Determine el dominio y la regla de correspondencia de las funciones $f - g$ y $\dfrac{f}{g}$

\vspace{-0.5cm}

\begin{center}
\[
\begin{array}{c@{\qquad}c}
  f(x) =
  \begin{cases}
    2x + 1, & x \in ]-5,3[ \\
    -x + 5, & x \in [3,10[
  \end{cases},  
  &
  g(x) =
  \begin{cases}
    1-x, & x \in [-10, 4] \\
    x-8, & x \in ]6, 15[
\end{cases}
\end{array}
\]
\end{center}

\vspace{0.5cm}

\begin{center}
\resizebox{\textwidth}{!}{%
\begin{tikzpicture}[yscale=1]
  \fill[gray!30] (-5,0) rectangle (3,0.5);   % overlap ]-5,3[
  \fill[gray!30] (3,0) rectangle (4,0.5);    % overlap [3,4]
  \fill[gray!30] (6,0) rectangle (10,0.5);   % overlap ]6,10[

  % Draw the number line
  \draw[thick] (-11,0) -- (16,0);

  % Tick marks and labels
  \foreach \x in {-10,-5,3,4,6,10,15}
    \draw[thick] (\x,0.08) -- (\x,-0.08);
  \foreach \x in {-10,-5,3,4,6,10,15}
    \node[below,font=\normalsize] at (\x,-0.08) {\x};

  % f(x) domain: ]-5,3[ and [3,10[
  \draw[blue,thick] (-5,1) -- (3,1);
  \draw[blue,thick] (3,1) -- (10,1);

  % Open/closed dots for f(x)
  \filldraw[white,draw=blue,thick] (-5,1) circle (1.5pt); % open
  \filldraw[white,draw=blue,thick] (3,1) circle (1.5pt); % open
  \filldraw[blue] (3,1) circle (1.5pt); % closed
  \filldraw[white,draw=blue,thick] (10,1) circle (1.5pt); % open

  % g(x) domain: [-10,4] and ]6,15[
  \draw[red,thick] (-10,0.5) -- (4,0.5);
  \draw[red,thick] (6,0.5) -- (15,0.5);

  % Open/closed dots for g(x)
  \filldraw[red] (-10,0.5) circle (1.5pt); % closed
  \filldraw[red] (4,0.5) circle (1.5pt); % closed
  \filldraw[white,draw=red,thick] (6,0.5) circle (1.5pt); % open
  \filldraw[white,draw=red,thick] (15,0.5) circle (1.5pt); % open

  % Legends
  \node[blue,right,font=\normalsize] at (10.5,1) {$f(x)$};
  \node[red,right,font=\normalsize] at (15.5,0.5) {$g(x)$};
\end{tikzpicture}}
\end{center}

\vspace{-0.5cm}

\[
\begin{minipage}{0.45\textwidth}
\[
\begin{array}{l}
  \multicolumn{1}{c}{f-g} \\[18pt]
  x \in ]-5, 3[ \; \Rightarrow \; 2x + 1 - (1 - x) = 3x \\[18pt]
  x \in [3, 4] \; \Rightarrow \; -x + 5 - 1 + x = 4 \\[18pt]
  x \in ]6, 10[ \; \Rightarrow \; -x + 5 - (x - 8) = 13 - 2x
\end{array}
\]
\end{minipage}
\hfill
\begin{minipage}{0.45\textwidth}
\[
\begin{array}{l}
  \multicolumn{1}{c}{f / g} \\[18pt]
  x \in ]-5, 3[ - \{1\}\; = \; \dfrac{2x + 1}{1 - x} \\[18pt]
  x \in [3, 4]  - \{1\}\; = \; \dfrac{- x + 5}{1 - x} \\[18pt]
  x \in ]6, 10[ - \{8\}\; = \; \dfrac{- x + 5}{x - 8}
\end{array}
\]
\end{minipage}
\]




\vspace{1cm}



\section*{Ejercicio 3}
Grafique la siguiente función $f(x) = -10 \cos\left(\dfrac{\pi}{6}x\right) + 4$, para un solo periodo.

\[
\begin{array}{c c c c c c}
  \left|a\right| = -10 \quad & w = \dfrac{\pi}{6} \quad & \O = 0 \quad & T = \dfrac{2\pi}{w} = 12 \quad & s = 3 \quad & b = 4
\end{array}
\]

\[
\begin{minipage}{0.3\textwidth}
\[
\begin{array}{c|c}
  0  & -6 \\ \hline
  3  & -4 \\ \hline
  6  & -6 \\ \hline
  9  & -4 \\ \hline
  12 & -6
\end{array}
\]
\end{minipage}
\hfill
\begin{minipage}{0.65\textwidth}
\begin{tikzpicture}
  \begin{axis}[
      axis lines=middle,
      xlabel={$x$},
      ylabel={$f(x)$},
      xtick={0,3,6,9,12},
      ytick={-6,-4,-2,0,2,4,6,8,10,12,14},
      ymin=-8, ymax=16,
      xmin=-1, xmax=13,
      grid=both,
      width=10cm,
      height=8cm,
      domain=0:12,
      samples=100,
    ]
    \addplot[blue, thick] {-10*cos(deg((pi/6)*x)) + 4};

    \addplot[blue, only marks, mark=*] coordinates {
      (0,-6)
      (3,4)
      (6,14)
      (9,4)
      (12,-6)
    };
  \end{axis}
\end{tikzpicture}
\end{minipage}
\]


\end{document}